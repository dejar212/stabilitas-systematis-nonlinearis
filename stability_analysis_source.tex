\documentclass[12pt,a4paper]{article}
\usepackage[utf8]{inputenc}
\usepackage[russian]{babel}
\usepackage{amsmath}
\usepackage{amsfonts}
\usepackage{amssymb}
\usepackage{graphicx}
\usepackage{geometry}
\usepackage{float}
\usepackage{caption}

\geometry{left=2cm,right=2cm,top=2cm,bottom=2cm}

\title{Качественный анализ областей устойчивости нелинейной системы управления методом функций Ляпунова}
\author{Сороколетов Денис Сергеевич}
\date{\today}

\begin{document}

\maketitle

\section{Введение и модель}

\subsection{Параметры системы}

В работе рассматривается система с параметрами, приведенными в Таблице \ref{tab:params}.

\begin{table}[H]
\centering
\begin{tabular}{|c|c|c|}
\hline
Параметр & Значение & Единица \\
\hline
$m_0$ (масса тележки) & 1.5 & кг \\
$m_1$ (масса звена 1) & 0.5 & кг \\
$m_2$ (масса звена 2) & 0.75 & кг \\
$L_1$ (длина звена 1) & 0.5 & м \\
$L_2$ (длина звена 2) & 0.75 & м \\
$g$ (ускорение св. падения) & 9.81 & м/с² \\
\hline
\end{tabular}
\caption{Физические параметры системы}
\label{tab:params}
\end{table}

\subsection{Уравнения движения}

Используя формализм Лагранжа, уравнения движения записываются в виде:

\begin{equation}
D(\theta)\ddot{\theta} + C(\theta, \dot{\theta})\dot{\theta} + G(\theta) = H u
\end{equation}

где $\theta = [\theta_0, \theta_1, \theta_2]^T$.

Матрица инерции $D(\theta)$:
\begin{equation}
D(\theta) = \begin{pmatrix} 
m_0 + m_1 + m_2 & (m_1/2 + m_2)L_1 \cos\theta_1 & m_2 (L_2/2) \cos\theta_2 \\
(m_1/2 + m_2)L_1 \cos\theta_1 & m_1 L_1^2/3 + m_2 L_1^2 & m_2 L_1 (L_2/2) \cos(\theta_1 - \theta_2) \\
m_2 (L_2/2) \cos\theta_2 & m_2 L_1 (L_2/2) \cos(\theta_1 - \theta_2) & m_2 L_2^2/3
\end{pmatrix}
\end{equation}

Вектор гравитационных сил $G(\theta)$:
\begin{equation}
G(\theta) = \begin{pmatrix} 
0 \\
-(m_1/2 + m_2)g L_1 \sin\theta_1 \\
-m_2 g (L_2/2) \sin\theta_2
\end{pmatrix}
\end{equation}

Вектор входных воздействий $H$, определяющий влияние горизонтальной силы $u$:
\begin{equation}
H = \begin{pmatrix} 1 \\ 0 \\ 0 \end{pmatrix}
\end{equation}

Матрица кориолисовых сил $C(\theta, \dot{\theta})$ определяется через символы Кристоффеля первого рода.

\subsection{Линеаризация и устойчивость}

Линеаризуем систему в окрестности верхнего неустойчивого равновесия $x_{eq} = 0$. Уравнения в пространстве состояний:
\begin{equation}
\dot{x} = Ax + Bu, \quad A = \frac{\partial f}{\partial x}\bigg|_{x=0}
\end{equation}

Матрица $A$ имеет блочную структуру:
\begin{equation}
A = \begin{pmatrix}
0_{3\times 3} & I_{3\times 3} \\
-D^{-1}(0)\frac{\partial G}{\partial \theta}\big|_{0} & 0_{3\times 3}
\end{pmatrix}
\end{equation}

Анализ собственных чисел матрицы $A$ (разомкнутая система) показывает наличие положительных действительных корней, что подтверждает неустойчивость системы без управления.

Для замкнутой системы с регулятором $u = -Kx$ матрица динамики $A_{cl} = A - BK$. Все собственные числа $\lambda(A_{cl})$ лежат в левой полуплоскости ($\text{Re}(\lambda) < 0$). Характеристический полином $\det(sI - A_{cl})$ удовлетворяет критерию Гурвица (все миноры матрицы Гурвица положительны), что гарантирует асимптотическую устойчивость линейного приближения.

\section{Анализ устойчивости методом Ляпунова}

\subsection{Функция Ляпунова}

В качестве кандидата на функцию Ляпунова выберем полную энергию системы относительно положения равновесия:
\begin{equation}
V(\theta, \dot{\theta}) = \frac{1}{2}\dot{\theta}^T D(\theta) \dot{\theta} + P(\theta)
\end{equation}
где $P(\theta)$ — потенциальная энергия, нормированная так, что $P(0)=0$.

Производная функции Ляпунова вдоль траекторий системы:
\begin{equation}
\dot{V} = \dot{\theta}^T H u
\end{equation}

При использовании линейного регулятора $u = -Kx$ в малой окрестности равновесия:
\begin{equation}
\dot{V} \approx -\dot{\theta}^T H K \begin{pmatrix} \theta \\ \dot{\theta} \end{pmatrix}
\end{equation}
Для оптимального LQR-регулятора, минимизирующего квадратичный функционал, можно показать, что в линейном приближении $\dot{V} = -x^T (Q + K^T R K) x < 0$, что обеспечивает диссипацию энергии и сходимость к равновесию.

На Рис. \ref{fig:lyap} показана эволюция полной энергии системы. Видно монотонное убывание, подтверждающее условие $\dot{V} \leq 0$.

\begin{figure}[H]
\centering
\includegraphics[width=0.8\textwidth]{figures/group2_lyapunov.pdf}
\caption{Эволюция полной энергии системы $V(t)$.}
\label{fig:lyap}
\end{figure}

\subsection{Аналитическая оценка области притяжения}

Оценим максимальное начальное отклонение $\theta_{max}$, при котором регулятор способен поглотить избыточную потенциальную энергию.
Потенциальная энергия при отклонении на угол $\theta$:
\begin{equation}
P(\theta) \approx \frac{1}{2} M_{eff} g L_{eff} \theta^2
\end{equation}
Приравнивая её к энергетическому ресурсу привода, получаем теоретическую оценку критического угла $\theta_{crit} \approx 35^\circ - 40^\circ$, что хорошо согласуется с результатами численного моделирования.

\subsection{Фазовый анализ}

Фазовые портреты (Рис. \ref{fig:phase1} и \ref{fig:phase2}) демонстрируют поведение системы при различных начальных отклонениях.

\begin{figure}[H]
\centering
\includegraphics[width=0.8\textwidth]{figures/group2_phase_portrait.pdf}
\caption{Фазовый портрет первого звена ($\theta_1, \dot{\theta}_1$). Зеленые траектории — устойчивые, красные — неустойчивые.}
\label{fig:phase1}
\end{figure}

\begin{figure}[H]
\centering
\includegraphics[width=0.8\textwidth]{figures/group2_phase_th2.pdf}
\caption{Фазовый портрет второго звена ($\theta_2, \dot{\theta}_2$).}
\label{fig:phase2}
\end{figure}

Сепаратриса, разделяющая области устойчивости и неустойчивости, проходит в районе $\theta_1 \approx 0.7$ рад ($40^\circ$).

\section{Исследование робастности}

Исследована чувствительность системы к вариациям параметров модели (массы, длины). На Рис. \ref{fig:robust} показана зависимость критического угла устойчивости от массы второго звена $m_2$.

\begin{figure}[H]
\centering
\includegraphics[width=0.8\textwidth]{figures/group2_robustness.pdf}
\caption{Зависимость критического угла отклонения от вариации массы $m_2$.}
\label{fig:robust}
\end{figure}

Анализ показывает, что увеличение массы второго звена на 20\% приводит к сужению области устойчивости примерно на $5^\circ$. Система демонстрирует достаточную робастность к малым возмущениям параметров.

\section{Заключение}

1. Линеаризация системы подтвердила неустойчивость разомкнутого объекта и возможность стабилизации линейным регулятором (выполнение критерия Гурвица).
2. Метод функций Ляпунова обосновал глобальную устойчивость в ограниченной области фазового пространства.
3. Численно и аналитически установлена граница области притяжения $\approx 40^\circ$.
4. Анализ робастности показал, что система сохраняет работоспособность при вариации параметров до $\pm 20\%$, однако критический угол при этом изменяется.

\end{document}
